\chapter{Speech to text}
\ifpdf
    \graphicspath{{Chapter3/Chapter3Figs/PNG/}{Chapter3/Chapter3Figs/PDF/}{Chapter3/Chapter3Figs/}}
\else
    \graphicspath{{Chapter3/Chapter3Figs/EPS/}{Chapter3/Chapter3Figs/}}
\fi

\section{Tổng quan}
\section{Mô hình hoạt động}
\section{Ứng dụng}
\section{Các vấn đề cần giải quyết}
\subsection{Dò tìm keyword}
\subsubsection{Yêu cầu}
Hoạt động liên tục -> nên hoạt động offline
Xử lý nhanh từng frame của âm thanh
Độ chính xác khá cao.
\subsubsection{Giải pháp}
sử dụng thư viên pocketsphinx
\subsection{Chuyển đổi lệnh người dùng thành văn bản}
\subsubsection{Yêu cầu}
Chỉ hoạt động khi người dùng ra lệnh
Độ chính xác rất cao.
\subsubsection{Giải pháp}
sử dụng thư viên google speech to text
\section{Thư viện pocketsphinx}
\subsection{Tổng quan}
\subsection{Chức năng}
\subsection{Cách cài đặt}
\subsection{Cách sử dụng}
các thông số của thư viện
\subsection{Ưu, nhược điểm}
Ưu: hỗ trợ offline, dò keyword
Nhược: độ chính xác kém
\subsection{Ứng dụng}
Module wake up, giúp kích hoạt hệ thống khi người dùng gọi wakup word
\section{Thư viện Google Speech To Text}
\subsection{Tổng quan}
\subsection{Chức năng}
\subsection{Cách cài đặt}
\subsection{Cách sử dụng}
Các thông số của thư viện
\subsection{Ưu, nhược điểm}
Ưu: độ chính xác cao
Nhược: yêu cầu internet, phải gửi toàn bộ file âm thanh 1 lúc, không stream được
\subsection{Ứng dụng}
Module TTS, giúp chuyển lệnh người dùng thành văn bản.