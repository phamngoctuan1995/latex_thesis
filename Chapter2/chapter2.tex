\chapter{Tín hiệu âm thanh, tiếng nói. Thư viện PyAudio}
\ifpdf
    \graphicspath{{Chapter2/Chapter2Figs/PNG/}{Chapter2/Chapter2Figs/PDF/}{Chapter2/Chapter2Figs/}}
\else
    \graphicspath{{Chapter2/Chapter2Figs/EPS/}{Chapter2/Chapter2Figs/}}
\fi

\section{Tổng quan về âm thanh, tiếng nói}
Âm thanh đóng vai trò quan trọng trọng cuộc sống con người

Âm thanh có thể được cảm nhận bởi con người thông qua thính giác

Một trong những dạng của âm thanh là tiếng nói

Tiếng nói đóng vai trò quan trọng trong hoạt động giao tiếp

Giao tiếp bằng tiếng nói là hoạt động giao tiếp nhanh, phổ biến, tiện lợi nhất

Âm thanh được nghiên cứu và ứng dụng trong nhiều lĩnh vực
\section{Các khái niệm cơ bản của âm thanh, tiếng nói}
wiki: https://en.wikipedia.org/wiki/Sound
\section{Cách lưu trữ âm thanh trong máy tính}
Âm thanh trong tự nhiên có dạng liên tục 

Máy tính lưu trữ dạng rời rạc -> cần lấy mẫu âm thanh theo 1 tần số.
\subsection{Các thông số của âm thanh khi lưu trữ trên máy tính}
sample rate

bitdepth

chanel
...
\subsection{Lưu trữ không nén}
file wav
\subsection{Lưu trữ nén}
file mp3

\section{Ứng dụng}
Âm thanh là công cụ tương tác giữa người sử dụng và ứng dụng.

Người sử dụng sử dụng tiếng nói để ra lệnh

Ứng dụng dùng tiếng nói để phản hồi

\section{Thư viện PyAudio}
\subsection{Tổng quan}
Là thư viện viết bằng Python hỗ trợ tất cả các Hđh

Hỗ trợ người dùng tương tác với âm thanh trên máy tính dễ dàng.

https://people.csail.mit.edu/hubert/pyaudio/docs/
\subsection{Chức năng}
Thu âm từ microphone của máy tính dưới dạng dữ liệu thô

Phát âm thanh ra loa của máy tính từ dữ liệu thô
\subsection{Cài đặt}
pip

build từ source
\subsection{Cách sử dụng}
2 cách sử dụng: blocking, non-blocking

các thông số của ứng dụng: samplerate, bitdeptg, chanel, ...
\subsection{Các ưu, khuyết điểm}
Ưu: hỗ trợ nhiều hđh, cài đặt đơn giản, nhẹ, dễ sử dụng.
Nhược: ít tính năng, chưa hỗ trợ phát từ file mp3, chưa hỗ trợ thu từ nhiều micro
\subsection{Ứng dụng}
PyAudio được sử dụng trong module Microphone của ứng dụng. Giúp thu âm và chuyển cho các module khác để xử lý.





