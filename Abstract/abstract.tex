\newpage
\chapter*{TÓM TẮT KHÓA LUẬN}
\addcontentsline{toc}{chapter}{TÓM TẮT KHÓA LUẬN} 

Trong xu hướng công nghệ hiện nay, vai trò của các trợ lý ảo ngày càng trở nên quan trọng. Các hãng công nghệ lớn thay nhau tung ra những trợ lý ảo của riêng mình tích hợp trên các thiết bị di động: Siri của Apple, Cortana của Microsoft, Google Assistant của Google, Alexa của Amazon,... Chức năng của các trợ lý ảo này ngày càng được mở rộng, từ những chức năng đơn giản như tra cứu, hỏi đáp, đến những chức năng cao hơn như quản lý lịch, gọi điện thoại, dẫn đường, điều khiển các thiết bị khác,... Khóa luận này có mục đích tạo ra một trợ lý ảo có khả năng chạy được trên nhiều nền tảng hệ điều hành khác nhau trên máy tính cá nhân. 

Nhận diện giọng nói là một trong những thành phần quan trọng nhất của một trợ lý ảo. Nhiều công ty và nhóm nghiên cứu lớn nhỏ đã nghiên cứu và đưa ra các bộ toolkit cũng như API cho việc nhận diện giọng nói, trong đó một trong những API có chất lượng được đánh giá tốt nhất là Google Speech API của gã khổng lồ công nghệ Google. Do đó, chúng tôi muốn tận dụng chất lượng của Google Speech API để tạo nên một trợ lý ảo có độ chính xác cao về nhận diện giọng nói. 

Kết quả sơ bộ mà khóa luận đạt được là tạo ra một trợ lý ảo có thể chạy trên các hệ điều hành phổ biến trên máy tính cá nhân như Windows, Linux, Mac. Trợ lý ảo có những chức năng cơ bản của một trợ lý ảo như hỏi đáp, tra cứu thông tin, trả lời các câu hỏi về thời gian, thời tiết, ngoài ra còn có thể phát nhạc theo yêu cầu và chào hỏi ở mức độ đơn giản. 

