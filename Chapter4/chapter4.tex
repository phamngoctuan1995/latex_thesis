\chapter{Text To Speech}
\ifpdf
    \graphicspath{{Chapter4/Chapter4Figs/PNG/}{Chapter4/Chapter4Figs/PDF/}{Chapter4/Chapter4Figs/}}
\else
    \graphicspath{{Chapter4/Chapter4Figs/EPS/}{Chapter4/Chapter4Figs/}}
\fi
\textit{Nội dung chương 4 sẽ giới thiệu tổng quan về bài toán Text to Speech, mô hình hoạt động, các ứng dụng của Text to Speech, các vấn đề cần giải quyết của module Text to Speech trong một hệ thống trợ lý ảo và cách giải quyết các vấn đề đó. Chương 4 cũng sẽ giới thiệu về chức năng, cách cài đặt, cách sử dụng cũng như các ưu nhược điểm của các thư viện iSpeech và Google Text to Speech.}
\section{Tổng quan}
Text-to-speech (TTS) gọi nôm na là tính năng chuyển văn bản thành giọng nói, được phát triển từ rất lâu trước đây. Có thể nói khởi nguồn của nó là năm 1779 khi mà nhà khoa học người Đan Mạch Christian Kratzenstein, lúc đó làm việc tại Viện Hàn lâm Khoa học Nga, xây dựng một mô hình có thể bắt chước giọng nói người với năm nguyên âm. Từ đó đến nay, qua nhiều giai đoạn phát triển thì công nghệ tổng hợp giọng nói đã có những bước phát triển vượt bậc, giọng nói đã tự nhiên hơn, dễ nghe hơn. Tuỳ vào công nghệ của từng nhà phát triển mà cho ra những giọng nói với chất lượng khác nhau.

Tham khảo thêm: Wiki - Tổng hợp giọng nói (en) (vi)​

Speech synthesis is the artificial production of human speech. A computer system used for this purpose is called a speech computer or speech synthesizer, and can be implemented in software or hardware products. A text-to-speech (TTS) system converts normal language text into speech; other systems render symbolic linguistic representations like phonetic transcriptions into speech.[1]

Synthesized speech can be created by concatenating pieces of recorded speech that are stored in a database. Systems differ in the size of the stored speech units; a system that stores phones or diphones provides the largest output range, but may lack clarity. For specific usage domains, the storage of entire words or sentences allows for high-quality output. Alternatively, a synthesizer can incorporate a model of the vocal tract and other human voice characteristics to create a completely "synthetic" voice output.[2]

The quality of a speech synthesizer is judged by its similarity to the human voice and by its ability to be understood clearly. An intelligible text-to-speech program allows people with visual impairments or reading disabilities to listen to written words on a home computer. Many computer operating systems have included speech synthesizers since the early 1990s.
\section{Mô hình hoạt động}
A text-to-speech system (or "engine") is composed of two parts:[3] a front-end and a back-end. The front-end has two major tasks. First, it converts raw text containing symbols like numbers and abbreviations into the equivalent of written-out words. This process is often called text normalization, pre-processing, or tokenization. The front-end then assigns phonetic transcriptions to each word, and divides and marks the text into prosodic units, like phrases, clauses, and sentences. The process of assigning phonetic transcriptions to words is called text-to-phoneme or grapheme-to-phoneme conversion. Phonetic transcriptions and prosody information together make up the symbolic linguistic representation that is output by the front-end. The back-end—often referred to as the synthesizer—then converts the symbolic linguistic representation into sound. In certain systems, this part includes the computation of the target prosody (pitch contour, phoneme durations),[4] which is then imposed on the output speech.
\section{Ứng dụng}
Các hệ thống này có nhiều ứng dụng. Ví dụ như hệ thống này có thể giúp người có thị lực kém (hoặc khiếm thị) nghe được máy đọc ra văn bản; đặc biệt là các văn bản có thể xử lý trên máy tính. Hệ thống như vậy có thể lắp đặt trong phần mềm xử lý văn bản hay trình duyệt mạng.
\section{Google Text To Speech}
\subsection{Tổng quan}
\subsection{Chức năng}
\subsection{Cách cài đặt}
\subsection{Cách sử dụng}
\subsection{Ưu, nhược điểm}
\subsection{Ứng dụng}
\section{iSpeech}
\subsection{Tổng quan}
\subsection{Chức năng}
\subsection{Cách cài đặt}
\subsection{Cách sử dụng}
\subsection{Ưu, nhược điểm}
\subsection{Ứng dụng}