% \pagebreak[4]
% \hspace*{1cm}
% \pagebreak[4]
% \hspace*{1cm}
% \pagebreak[4]

\chapter{Mở Đầu }
\textit{Nội dung của chương 1 giới thiệu tổng quan về đề tài, nêu ra mục tiêu của khóa luận, và cấu trúc nội dung của luận văn.}
\ifpdf
    \graphicspath{{Chapter1/Chapter1Figs/PNG/}{Chapter1/Chapter1Figs/PDF/}{Chapter1/Chapter1Figs/}}
\else
    \graphicspath{{Chapter1/Chapter1Figs/EPS/}{Chapter1/Chapter1Figs/}}
\fi

\section{Tổng quan về đề tài}

Trợ lý ảo là một phần mềm trên máy tính hoặc thiết bị di động có khả năng hỗ trợ người dùng thực hiện nhiều loại công việc, nhận lệnh từ người dùng dưới dạng ngôn ngữ tự nhiên, thường là giọng nói. Nhờ khả năng nhận lệnh và phản hồi qua giọng nói, người dùng có thể ra lệnh cho trợ lý ảo mà không cần phải thao tác bằng tay trên thiết bị.

Trong xu hướng công nghệ ngày càng tiên tiến, việc sở hữu một trợ lý ảo sẽ giúp cho người dùng có những trải nghiệm mới mẻ và thú vị hơn khi sử dụng các thiết bị công nghệ nhờ vào sự tiện dụng, mạnh mẽ với nhiều chức năng đa dạng, cũng như tính tự nhiên trong giao tiếp giữa người và máy. Khi sử dụng các trợ lý ảo tiên tiến nhất hiện nay, người dùng sẽ có cảm giác được giao tiếp với một người trợ lý thực sự chứ không phải là một cái máy. Số lượng chức năng của các trợ lý ảo ngày càng tăng, từ những chức năng cơ bản như hỏi đáp, tra cứu, tìm kiếm thông tin, đến những chức năng nâng cao hơn như quản lý lịch, quản lý email, thực hiện cuộc gọi, gửi tin nhắn, điều khiển các thiết bị trong nhà, và thậm chí là đặt chỗ nhà hàng!

...

\section{Mục tiêu của khóa luận}

...

\section{Nội dung luận văn}

...
